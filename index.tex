% Options for packages loaded elsewhere
% Options for packages loaded elsewhere
\PassOptionsToPackage{unicode}{hyperref}
\PassOptionsToPackage{hyphens}{url}
\PassOptionsToPackage{dvipsnames,svgnames,x11names}{xcolor}
%
\documentclass[
  letterpaper,
  DIV=11,
  numbers=noendperiod]{scrreport}
\usepackage{xcolor}
\usepackage{amsmath,amssymb}
\setcounter{secnumdepth}{-\maxdimen} % remove section numbering
\usepackage{iftex}
\ifPDFTeX
  \usepackage[T1]{fontenc}
  \usepackage[utf8]{inputenc}
  \usepackage{textcomp} % provide euro and other symbols
\else % if luatex or xetex
  \usepackage{unicode-math} % this also loads fontspec
  \defaultfontfeatures{Scale=MatchLowercase}
  \defaultfontfeatures[\rmfamily]{Ligatures=TeX,Scale=1}
\fi
\usepackage{lmodern}
\ifPDFTeX\else
  % xetex/luatex font selection
  \setmainfont[]{D2Coding ligature:style=Regular}
\fi
% Use upquote if available, for straight quotes in verbatim environments
\IfFileExists{upquote.sty}{\usepackage{upquote}}{}
\IfFileExists{microtype.sty}{% use microtype if available
  \usepackage[]{microtype}
  \UseMicrotypeSet[protrusion]{basicmath} % disable protrusion for tt fonts
}{}
\makeatletter
\@ifundefined{KOMAClassName}{% if non-KOMA class
  \IfFileExists{parskip.sty}{%
    \usepackage{parskip}
  }{% else
    \setlength{\parindent}{0pt}
    \setlength{\parskip}{6pt plus 2pt minus 1pt}}
}{% if KOMA class
  \KOMAoptions{parskip=half}}
\makeatother
% Make \paragraph and \subparagraph free-standing
\makeatletter
\ifx\paragraph\undefined\else
  \let\oldparagraph\paragraph
  \renewcommand{\paragraph}{
    \@ifstar
      \xxxParagraphStar
      \xxxParagraphNoStar
  }
  \newcommand{\xxxParagraphStar}[1]{\oldparagraph*{#1}\mbox{}}
  \newcommand{\xxxParagraphNoStar}[1]{\oldparagraph{#1}\mbox{}}
\fi
\ifx\subparagraph\undefined\else
  \let\oldsubparagraph\subparagraph
  \renewcommand{\subparagraph}{
    \@ifstar
      \xxxSubParagraphStar
      \xxxSubParagraphNoStar
  }
  \newcommand{\xxxSubParagraphStar}[1]{\oldsubparagraph*{#1}\mbox{}}
  \newcommand{\xxxSubParagraphNoStar}[1]{\oldsubparagraph{#1}\mbox{}}
\fi
\makeatother


\usepackage{longtable,booktabs,array}
\usepackage{calc} % for calculating minipage widths
% Correct order of tables after \paragraph or \subparagraph
\usepackage{etoolbox}
\makeatletter
\patchcmd\longtable{\par}{\if@noskipsec\mbox{}\fi\par}{}{}
\makeatother
% Allow footnotes in longtable head/foot
\IfFileExists{footnotehyper.sty}{\usepackage{footnotehyper}}{\usepackage{footnote}}
\makesavenoteenv{longtable}
\usepackage{graphicx}
\makeatletter
\newsavebox\pandoc@box
\newcommand*\pandocbounded[1]{% scales image to fit in text height/width
  \sbox\pandoc@box{#1}%
  \Gscale@div\@tempa{\textheight}{\dimexpr\ht\pandoc@box+\dp\pandoc@box\relax}%
  \Gscale@div\@tempb{\linewidth}{\wd\pandoc@box}%
  \ifdim\@tempb\p@<\@tempa\p@\let\@tempa\@tempb\fi% select the smaller of both
  \ifdim\@tempa\p@<\p@\scalebox{\@tempa}{\usebox\pandoc@box}%
  \else\usebox{\pandoc@box}%
  \fi%
}
% Set default figure placement to htbp
\def\fps@figure{htbp}
\makeatother


% definitions for citeproc citations
\NewDocumentCommand\citeproctext{}{}
\NewDocumentCommand\citeproc{mm}{%
  \begingroup\def\citeproctext{#2}\cite{#1}\endgroup}
\makeatletter
 % allow citations to break across lines
 \let\@cite@ofmt\@firstofone
 % avoid brackets around text for \cite:
 \def\@biblabel#1{}
 \def\@cite#1#2{{#1\if@tempswa , #2\fi}}
\makeatother
\newlength{\cslhangindent}
\setlength{\cslhangindent}{1.5em}
\newlength{\csllabelwidth}
\setlength{\csllabelwidth}{3em}
\newenvironment{CSLReferences}[2] % #1 hanging-indent, #2 entry-spacing
 {\begin{list}{}{%
  \setlength{\itemindent}{0pt}
  \setlength{\leftmargin}{0pt}
  \setlength{\parsep}{0pt}
  % turn on hanging indent if param 1 is 1
  \ifodd #1
   \setlength{\leftmargin}{\cslhangindent}
   \setlength{\itemindent}{-1\cslhangindent}
  \fi
  % set entry spacing
  \setlength{\itemsep}{#2\baselineskip}}}
 {\end{list}}
\usepackage{calc}
\newcommand{\CSLBlock}[1]{\hfill\break\parbox[t]{\linewidth}{\strut\ignorespaces#1\strut}}
\newcommand{\CSLLeftMargin}[1]{\parbox[t]{\csllabelwidth}{\strut#1\strut}}
\newcommand{\CSLRightInline}[1]{\parbox[t]{\linewidth - \csllabelwidth}{\strut#1\strut}}
\newcommand{\CSLIndent}[1]{\hspace{\cslhangindent}#1}



\setlength{\emergencystretch}{3em} % prevent overfull lines

\providecommand{\tightlist}{%
  \setlength{\itemsep}{0pt}\setlength{\parskip}{0pt}}



 


\KOMAoption{captions}{tableheading}
\makeatletter
\@ifpackageloaded{bookmark}{}{\usepackage{bookmark}}
\makeatother
\makeatletter
\@ifpackageloaded{caption}{}{\usepackage{caption}}
\AtBeginDocument{%
\ifdefined\contentsname
  \renewcommand*\contentsname{Table of contents}
\else
  \newcommand\contentsname{Table of contents}
\fi
\ifdefined\listfigurename
  \renewcommand*\listfigurename{List of Figures}
\else
  \newcommand\listfigurename{List of Figures}
\fi
\ifdefined\listtablename
  \renewcommand*\listtablename{List of Tables}
\else
  \newcommand\listtablename{List of Tables}
\fi
\ifdefined\figurename
  \renewcommand*\figurename{Figure}
\else
  \newcommand\figurename{Figure}
\fi
\ifdefined\tablename
  \renewcommand*\tablename{Table}
\else
  \newcommand\tablename{Table}
\fi
}
\@ifpackageloaded{float}{}{\usepackage{float}}
\floatstyle{ruled}
\@ifundefined{c@chapter}{\newfloat{codelisting}{h}{lop}}{\newfloat{codelisting}{h}{lop}[chapter]}
\floatname{codelisting}{Listing}
\newcommand*\listoflistings{\listof{codelisting}{List of Listings}}
\makeatother
\makeatletter
\makeatother
\makeatletter
\@ifpackageloaded{caption}{}{\usepackage{caption}}
\@ifpackageloaded{subcaption}{}{\usepackage{subcaption}}
\makeatother
\usepackage{bookmark}
\IfFileExists{xurl.sty}{\usepackage{xurl}}{} % add URL line breaks if available
\urlstyle{same}
\hypersetup{
  pdftitle={Extending Communities of Practice Through Digital Networks: A Relational Event Model Analysis of Knowledge Sharing Dynamics in an Online Professional Community},
  pdfauthor={Chungil Chae},
  colorlinks=true,
  linkcolor={blue},
  filecolor={Maroon},
  citecolor={Blue},
  urlcolor={Blue},
  pdfcreator={LaTeX via pandoc}}


\title{Extending Communities of Practice Through Digital Networks: A
Relational Event Model Analysis of Knowledge Sharing Dynamics in an
Online Professional Community}
\usepackage{etoolbox}
\makeatletter
\providecommand{\subtitle}[1]{% add subtitle to \maketitle
  \apptocmd{\@title}{\par {\large #1 \par}}{}{}
}
\makeatother
\subtitle{working}
\author{Chungil Chae}
\date{Mon, 24 November 2025}
\begin{document}
\maketitle

\renewcommand*\contentsname{Table of contents}
{
\hypersetup{linkcolor=}
\setcounter{tocdepth}{3}
\tableofcontents
}
\listoffigures
\listoftables

\bookmarksetup{startatroot}

\chapter*{Project Details}\label{project-details}
\addcontentsline{toc}{chapter}{Project Details}

\markboth{Project Details}{Project Details}

Updated in 2021-09-01

\section*{Citation}\label{citation}
\addcontentsline{toc}{section}{Citation}

\markright{Citation}

\section*{Abstract}\label{abstract}
\addcontentsline{toc}{section}{Abstract}

\markright{Abstract}

\begin{quote}
In today's hyper-connected environment, organizations increasingly rely
on distributed expertise and cross-boundary knowledge exchange to remain
competitive. While Communities of Practice (CoPs) have been widely
recognized as essential structures for collective learning, the rise of
digital technologies has enabled a new form of community---extended
Communities of Practice (eCoPs)---that transcend geographical,
organizational, and temporal boundaries. These digitally mediated, fluid
networks connect diverse practitioners and support continuous,
practice-oriented knowledge sharing. Despite their growing importance,
theoretical and methodological tools for understanding the temporal
dynamics of knowledge exchange in eCoPs remain limited. Existing
research often relies on static or aggregated network representations
that obscure the sequential, event-based structure of knowledge
interactions. To address this gap, we apply Relational Event Models
(REM), a statistical framework designed to analyze time-stamped
interaction sequences and capture how prior events shape subsequent
behavior. Using four months of interaction data from CloudBro, an online
professional community of IT practitioners, we examine 216 relational
events generated by 59 participants across 309 posts in 89 discussion
threads. By leveraging REM's temporal granularity, this study elucidates
how conversational sequences, reciprocity, and emergent attention
patterns structure knowledge flows in an extended CoP. The findings
contribute to theory development on digital knowledge communities and
offer practical insights for designing and managing effective eCoPs.
\end{quote}

\section*{Planning}\label{planning}
\addcontentsline{toc}{section}{Planning}

\markright{Planning}

\subsection*{Schedule}\label{schedule}
\addcontentsline{toc}{subsection}{Schedule}

\begin{itemize}
\tightlist
\item
  Nov, 2025

  \begin{itemize}
  \tightlist
  \item
    Draft
  \end{itemize}
\item
  Dec, 2025

  \begin{itemize}
  \tightlist
  \item
    Submit
  \end{itemize}
\end{itemize}

\subsection*{Progress}\label{progress}
\addcontentsline{toc}{subsection}{Progress}

\begin{itemize}
\tightlist
\item
  Pilot analysis done
\item
  writing
\end{itemize}

\section*{Authorship and Authors}\label{authorship-and-authors}
\addcontentsline{toc}{section}{Authorship and Authors}

\markright{Authorship and Authors}

\subsection*{Author Contribution}\label{author-contribution}
\addcontentsline{toc}{subsection}{Author Contribution}

\begin{itemize}
\tightlist
\item
  Supervision: Chungil (Chad) Chae
\item
  Project administration:

  \begin{itemize}
  \tightlist
  \item
    Chungil (Chad) Chae
  \end{itemize}
\item
  Conceptualization:

  \begin{itemize}
  \tightlist
  \item
    Chungil (Chad) Chae
  \end{itemize}
\item
  Data accusation:
\item
  Cloud Bro
\item
  Data curation:
\item
  Jiongcheng Lu (Tony)
\item
  Formal analysis:
\item
  Chungil (Chad) Chae
\item
  Jiongcheng Lu (Tony)
\item
  Investigation:

  \begin{itemize}
  \tightlist
  \item
    Jiongcheng Lu (Tony)
  \end{itemize}
\item
  Methodology:

  \begin{itemize}
  \tightlist
  \item
    Chungil (Chad) Chae
  \item
    Jiongcheng Lu (Tony)
  \end{itemize}
\item
  Resources:
\item
  Validation:
\item
  Visualization:

  \begin{itemize}
  \tightlist
  \item
    Jiongcheng Lu (Tony)
  \end{itemize}
\item
  Writing -- original draft:

  \begin{itemize}
  \tightlist
  \item
    Chungil (Chad) Chae
  \item
    Jiongcheng Lu (Tony)
  \end{itemize}
\item
  Writing -- review \& editing:

  \begin{itemize}
  \tightlist
  \item
    Chungil (Chad) Chae
  \end{itemize}
\end{itemize}

\subsection*{Authors}\label{authors}
\addcontentsline{toc}{subsection}{Authors}

\subsubsection*{Chungil Chae, Ph.D.,
M.S.}\label{chungil-chae-ph.d.-m.s.}
\addcontentsline{toc}{subsubsection}{Chungil Chae, Ph.D., M.S.}

\begin{itemize}
\tightlist
\item
  chadchae@gmail.com
\item
  orcid:
  \href{https://orcid.org/0000-0002-7364-1525}{000-0002-7364-1525}
\item
  \href{https://scholar.google.com/citations?user=c4lRBrkAAAAJ&hl=en}{Google
  Scholar}
\item
  Chae (2024)
\end{itemize}

\pandocbounded{\includegraphics[keepaspectratio]{img/chadchae.png}}

\begin{quote}
Chungil Chae (Chad) is an assistant professor in the field of business
analytics, with a distinguished track record in organizational behavior,
human resource development (HRD), learning, and development. With a
prolific publication record that spans various dimensions of HRD and
organizational studies, Dr.~Chae has made significant contributions to
understanding the dynamics of organizational support on knowledge
sharing, virtual team leadership, and the structural determinants of HRD
research collaboration networks. And his work embodies a deep commitment
to enhancing understanding and practices in organizational behavior,
HRD, learning, and development. His interdisciplinary research not only
contributes to academic discourse but also offers tangible strategies
for organizational improvement and individual development.
\end{quote}

\subsubsection*{Jiongcheng Lu (Tony)}\label{jiongcheng-lu-tony}
\addcontentsline{toc}{subsubsection}{Jiongcheng Lu (Tony)}

\begin{itemize}
\tightlist
\item
  lujiongcheng1@gmail.com
\item
  orcid:
\item
  google scholar:
\end{itemize}

\pandocbounded{\includegraphics[keepaspectratio]{img/tony.png}}

\begin{quote}
JJiongcheng Lu (Tony) is a Ph.D candidate. He hold a solid
interdisciplinary background that bridges industry experience with
academic research. He completed undergraduate studies at Wenzhou-Kean
University, a Sino--American joint institution, majoring in Finance and
minoring in Economics. During his studies, he served as a Teaching
Assistant for Python and R data analysis and programming, and worked as
a Research Assistant on projects involving data mining and healthcare
accessibility in collaboration with Kean University. He later completed
master's degree at University College London (UCL), where He focused on
the intersection of finance and the healthcare sector. After graduation,
he worked as a Research Assistant in Health Economics at Peking
University, further strengthening his quantitative and analytical
expertise. He is currently pursuing PhD at the University of Sydney.
Beyond academia, He have completed internships at banks, investment
institutions, and technology research institutes, gaining broad exposure
to finance, innovation, and applied research. He also founded own
technology startup, which provided hands-on entrepreneurial experience.
\end{quote}

\subsection*{Acknowledgement}\label{acknowledgement}
\addcontentsline{toc}{subsection}{Acknowledgement}

\begin{itemize}
\tightlist
\item
  tba
\end{itemize}

\section*{Declearation}\label{declearation}
\addcontentsline{toc}{section}{Declearation}

\markright{Declearation}

\subsection*{IRB}\label{irb}
\addcontentsline{toc}{subsection}{IRB}

\subsection*{Funding}\label{funding}
\addcontentsline{toc}{subsection}{Funding}

\subsection*{AI}\label{ai}
\addcontentsline{toc}{subsection}{AI}

\part{Research Logs}

\chapter*{Research Log}\label{research-log}
\addcontentsline{toc}{chapter}{Research Log}

\markboth{Research Log}{Research Log}

\section*{xxxx-xx-xx}\label{xxxx-xx-xx}
\addcontentsline{toc}{section}{xxxx-xx-xx}

\markright{xxxx-xx-xx}

\begin{itemize}
\tightlist
\item
  entry.
\end{itemize}

\chapter*{Meeting Log}\label{meeting-log}
\addcontentsline{toc}{chapter}{Meeting Log}

\markboth{Meeting Log}{Meeting Log}

\section*{xxxx-xx-xx}\label{xxxx-xx-xx-1}
\addcontentsline{toc}{section}{xxxx-xx-xx}

\markright{xxxx-xx-xx}

\begin{itemize}
\tightlist
\item[$\boxtimes$]
  10:00-10:30: Kick-off meeting
\end{itemize}

\chapter*{Analysis Version}\label{analysis-version}
\addcontentsline{toc}{chapter}{Analysis Version}

\markboth{Analysis Version}{Analysis Version}

\section*{Version 0.0.1}\label{version-0.0.1}
\addcontentsline{toc}{section}{Version 0.0.1}

\markright{Version 0.0.1}

\begin{itemize}
\tightlist
\item
  Starting draft
\end{itemize}

\chapter*{Draft Version}\label{draft-version}
\addcontentsline{toc}{chapter}{Draft Version}

\markboth{Draft Version}{Draft Version}

\section*{Ver 0.0.1}\label{ver-0.0.1}
\addcontentsline{toc}{section}{Ver 0.0.1}

\markright{Ver 0.0.1}

\begin{itemize}
\tightlist
\item
  Starting draft
\end{itemize}

\part{Research Note}

\chapter*{Ideas and Thoughts}\label{ideas-and-thoughts}
\addcontentsline{toc}{chapter}{Ideas and Thoughts}

\markboth{Ideas and Thoughts}{Ideas and Thoughts}

\chapter*{Research Q \& A}\label{research-q-a}
\addcontentsline{toc}{chapter}{Research Q \& A}

\markboth{Research Q \& A}{Research Q \& A}

\chapter*{Procedures}\label{procedures}
\addcontentsline{toc}{chapter}{Procedures}

\markboth{Procedures}{Procedures}

\part{Theoritical Framework}

\chapter*{Related Theories}\label{related-theories}
\addcontentsline{toc}{chapter}{Related Theories}

\markboth{Related Theories}{Related Theories}

\chapter*{Theortical Relationship}\label{theortical-relationship}
\addcontentsline{toc}{chapter}{Theortical Relationship}

\markboth{Theortical Relationship}{Theortical Relationship}

\chapter*{Theoritical Framework}\label{theoritical-framework-1}
\addcontentsline{toc}{chapter}{Theoritical Framework}

\markboth{Theoritical Framework}{Theoritical Framework}

\section*{Hypothesis}\label{hypothesis}
\addcontentsline{toc}{section}{Hypothesis}

\markright{Hypothesis}

\part{Methodology and Method}

\chapter*{Methdology}\label{methdology}
\addcontentsline{toc}{chapter}{Methdology}

\markboth{Methdology}{Methdology}

\chapter*{Method}\label{method}
\addcontentsline{toc}{chapter}{Method}

\markboth{Method}{Method}

\chapter*{Data Collection}\label{data-collection}
\addcontentsline{toc}{chapter}{Data Collection}

\markboth{Data Collection}{Data Collection}

\chapter*{Data}\label{data}
\addcontentsline{toc}{chapter}{Data}

\markboth{Data}{Data}

\part{Literature Review}

\chapter*{Searching and Inclusion \&
Exclusion}\label{searching-and-inclusion-exclusion}
\addcontentsline{toc}{chapter}{Searching and Inclusion \& Exclusion}

\markboth{Searching and Inclusion \& Exclusion}{Searching and Inclusion
\& Exclusion}

\chapter*{Round 1}\label{round-1}
\addcontentsline{toc}{chapter}{Round 1}

\markboth{Round 1}{Round 1}

\section*{Search kewwords and
category}\label{search-kewwords-and-category}
\addcontentsline{toc}{section}{Search kewwords and category}

\markright{Search kewwords and category}

\section*{keywords combination}\label{keywords-combination}
\addcontentsline{toc}{section}{keywords combination}

\markright{keywords combination}

\chapter*{Round 2}\label{round-2}
\addcontentsline{toc}{chapter}{Round 2}

\markboth{Round 2}{Round 2}

\section*{Search kewwords and
category}\label{search-kewwords-and-category-1}
\addcontentsline{toc}{section}{Search kewwords and category}

\markright{Search kewwords and category}

\section*{keywords combination}\label{keywords-combination-1}
\addcontentsline{toc}{section}{keywords combination}

\markright{keywords combination}

\chapter*{Round 3}\label{round-3}
\addcontentsline{toc}{chapter}{Round 3}

\markboth{Round 3}{Round 3}

\section*{Search kewwords and
category}\label{search-kewwords-and-category-2}
\addcontentsline{toc}{section}{Search kewwords and category}

\markright{Search kewwords and category}

\section*{keywords combination}\label{keywords-combination-2}
\addcontentsline{toc}{section}{keywords combination}

\markright{keywords combination}

\chapter*{PRISMA}\label{prisma}
\addcontentsline{toc}{chapter}{PRISMA}

\markboth{PRISMA}{PRISMA}

\chapter*{Reference List}\label{reference-list}
\addcontentsline{toc}{chapter}{Reference List}

\markboth{Reference List}{Reference List}

Category, Classification and Decision Note for Selected Literature in
Rounds

\section*{Round 1}\label{round-1-1}
\addcontentsline{toc}{section}{Round 1}

\markright{Round 1}

\section*{Round 2}\label{round-2-1}
\addcontentsline{toc}{section}{Round 2}

\markright{Round 2}

\section*{Round 3}\label{round-3-1}
\addcontentsline{toc}{section}{Round 3}

\markright{Round 3}

\section*{Additional (during and after
writing)}\label{additional-during-and-after-writing}
\addcontentsline{toc}{section}{Additional (during and after writing)}

\markright{Additional (during and after writing)}

\chapter*{Reserach Problems}\label{reserach-problems}
\addcontentsline{toc}{chapter}{Reserach Problems}

\markboth{Reserach Problems}{Reserach Problems}

\chapter*{Key References}\label{key-references}
\addcontentsline{toc}{chapter}{Key References}

\markboth{Key References}{Key References}

\chapter*{Quotes and Paraphrases}\label{quotes-and-paraphrases}
\addcontentsline{toc}{chapter}{Quotes and Paraphrases}

\markboth{Quotes and Paraphrases}{Quotes and Paraphrases}

\bookmarksetup{startatroot}

\chapter*{Products}\label{products}
\addcontentsline{toc}{chapter}{Products}

\markboth{Products}{Products}

\bookmarksetup{startatroot}

\chapter*{References}\label{references}
\addcontentsline{toc}{chapter}{References}

\markboth{References}{References}

\phantomsection\label{refs}
\begin{CSLReferences}{1}{0}
\bibitem[\citeproctext]{ref-chadchae}
Chae, C. (2024). \emph{Introduction to chad (chungil) chae}.
\url{https://chadchae.github.io}

\end{CSLReferences}




\end{document}
